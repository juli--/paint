\documentclass[a4paper,12pt]{article}
\usepackage[utf8]{inputenc}
\usepackage{newcent}
\usepackage[ngerman]{babel}
\usepackage[utf8]{inputenc}      
\usepackage{amsmath}		% wichtige symbole
\usepackage{bbm}	 	% mathbb
\usepackage{hyperref}
\usepackage{amsmath}%
\usepackage{MnSymbol}%
\usepackage{wasysym}%
 \usepackage{booktabs}
 \usepackage[table,xcdraw]{xcolor}
\usepackage{geometry}
\geometry{a4paper, top=25mm, left=25mm, right=25mm, bottom=30mm,
headsep=10mm, footskip=12mm}
\newcommand{\openEx}[1]{ {\bf Aufgabe #1}}
\newcommand{\closeEx}{\vspace{1.5cm}}

% Mathbb versions
\newcommand{\N}{\mathbbm N}
\newcommand{\K}{\mathbbm K}
\newcommand{\R}{ R}
\newcommand{\C}{\mathbbm C}
\newcommand{\Z}{\mathbbm Z}

% Theorem und Satz           
\newtheorem{thm}{Satz}[section]
\newtheorem{myDef}[thm]{Definition}
% \newtheorem{thm}{Satz}[section]
\newtheorem{ex}{Aufgabe}[section]
% \newtheorem{myDef}[thm]{Definition}
\newtheorem{myExmpl}[thm]{Beispiel}
\newtheorem{myCmmt}[thm]{Bemerkung}
\newtheorem{lem}[thm]{Lemma}
\newtheorem{prop}[thm]{Proposition}
\newtheorem{myKor}[thm]{Korollar}
\newtheorem{myLemma}[thm]{Lemma}
\makeatletter
\newcommand*{\eqnumref}[1]{%
	\begingroup
	\let\tagtext\@gobble
	\let\tagcomma\relax
	\eqref{#1}%
	\endgroup
}
\newcommand*{\eqtextref}[1]{%
	\begingroup
	\let\tagcomma\relax
	\let\tagnumber\@gobble
	\ref{#1}%
	\endgroup
} 

%\newcommand{\openEx}[1]{ {\bf Aufgabe #1}}
% \newcommand{\closeEx}{\vspace{1.5cm}}





\begin{document}


\begin{center}
\Large{\textbf{{Dokumentationsauszug}}}\\
\large{\textbf{{Paint}}}
\end{center}

~\\\begin{tabular}[h]{@{}p{0.82\textwidth} @{}p{0.8\textwidth}}
\begin{tabular}{@{}l}
Julius Hülsmann
\end{tabular}
&\begin{tabular}{@{}r}
%Vortrag AL II\\
18.05.2015 \\
\textbf{Status}: in Planung
\end{tabular}
\end{tabular}
\vspace{0.5cm}
\normalsize



\section{Schriftsuche}
\begin{myDef}[\textsc{Buchstabensegment}]~\\
	Bezeichne als \textsc{Buchstabensegment} ein Tupel $(f, Z)$ einer abschnittsweise definierten, stetigen Funktion 
	\begin{align}
		f: I &\rightarrow \R^2 \nonumber \quad mit~ f(t) = (x, y)\nonumber  \\
		f_i: [t_i, t_{i+1}] &\rightarrow \R^2 \nonumber \quad \forall i = 1, \dotsc, n \\
		f_{|[t_i, t_{i+1}]} &:= f_i \nonumber
	\end{align} zusammen mit einer beliebigen und beliebig groben Unterteilung $$Z := \{0 = t_0 <= t_1 <= \dotsc <= t_n\}$$ eines beliebigen Invervalls $I := [0, t_n] \subset \R$, für das die folgende Bedingung gelte:
		$$h_i :=  t_{i+1} - t_{i} = \| f_i(t_{i+1}) - f_i(t_i) \|_2 \quad \forall i = 1, \dotsc, n.$$
\end{myDef}

\begin{myDef}[\textsc{Position eines Buchstabensegmentes}]~\\
	Als \textsc{Position des Buchstabensegmentes} $(f, Z)$ bezeichne $f(t_0) \in \R^2$.
\end{myDef}

\begin{myDef}[\textsc{Übereinstimmungsfunktion}]~\\
	Eine Funktion $g((f_1, Z_1), (f_2, Z_2)) = g((f_2, Z_2), (f_1, Z_1)) = p$ , die als Übergabeparameter zwei \textsc{Buchstabensegmente} entgegennimmt und eine Wahrscheinlichkeits- \\übereinstimmung zurückgibt, wird im Folgenden als
	 \textsc{Übereinstimmungsfunktion} bezeichnet. 
\end{myDef}

\paragraph{Aufgabenstellung}

Es ist gegeben:
\begin{enumerate}
	\item[a)] Der gesuchte Ausdruck als Liste von  \textsc{Buchstabensegmenten},
	\item[b)] Der zu durchsuchende Ausdruck als Liste von  \textsc{Buchstabensegmenten},
	\item[c)] Eine angegebene Wahrscheinlichkeit $p \in [0, 1]$.
\end{enumerate} 
Ziel ist es, alle zusammenhängenden Teilausdrücke des zu durchsuchenden Ausdrucks  zu lokalisieren, die zu einer Wahrscheinlichkeit $\geq p$ mit dem gesuchten Ausdruck übereinstimmt.
Die \textsc{Übereinstimmungsfunktion} $g$ muss folgenden Kriterien genügen:
\begin{enumerate}
	\item[a)] Positionsinvarianz, also $g((f_1 + c, I_1), (f_2, I_2 )) = g((f_1, I_1), (f_2, I_2 ))  \forall c \in \R^2$
	\item[b)] Verschiebungsinvarianz, also wenn $f_{1|I_2 + t} == f_{2|I_2}$ mit $I_2 \subseteq I_1, t \in \R $ soll gelten $g((f_1, I_1), (f_2, I_2)) = 1$
	\item[b)] Skalierungsinvarianz
	\item[c)] Rotationsinvarizanz
	\item[d)] Scherungsinvarianz
	\item[e)] Separationsinvarianz (Aufteilung in verschiedene PaintObjects.)
\end{enumerate}
Bemerkung: An eine Streckunginvarianz bezgülgich der oberen Intervallgrenze $t_n$ muss hier nicht extra gedacht werden, da $t_n$ = der Norm der gezeichneten Kurve ist.
Allerdings düfte das in die Skalierungsinvarianz eingehen.

\subsection{Transformation des gesuchten Ausdrucks}
- Aufteilung in "Wörter", welche Energiesparend in ein PaintObjectWriting zusammengeführt werden.
- Suche nach einzelnen Wörtern.
- Ordnung der Fundstellen nach ungefähr höchster Übereinstimmung
()Überprüfen, ob Gesamtausdruck gefunden wurde)
	-> JA 		=> Gesamtausdruck hat beste
	-> NEIN   => länsten zusammenhängenden gefundenen Ausdruck angeben.
	
	falls mehrere Objekte ungefähr die gleiche Übereinstimmung besitzen, Ordnen nach Position y.
	



Transformation:
jedes PaintObjectWriting kann transformiert werden in eine abschnittsweise definierte Funktion $f: \R \Rightarrow R^2 $ mit $f(t) = (x, y)$


\end{document}